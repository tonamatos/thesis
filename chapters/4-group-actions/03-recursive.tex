% /chapters/4-group-actions/03-recursive.tex

\section{Recursive constructions}\label{sec:recursive_constructions}

\subsection{Feasible edge-replacements}\label{subsec:fer}

Consider this combinatorial puzzle:
One is given a graph.
Each time, one may remove an edge and add it wherever there is no edge (trivially, one can add it back to its original position), but the new graph must be isomorphic to the previous graph.
Such an action is called a \emph{feasible edge-replacement}, which is formalized after these examples.
Can one attain every possible configuration of the graph by repeateadly changing edges like this?
It is clear that the answer depends on the structure of the particular starting graph.
For instance, a cycle of four edges has no valid non-trivial movements since, having removed an edge, the only possibility that preserves the cycle is putting the edge back.
Graphically, Figure~\ref{fig:4_cycles} depicts two isomorphic graphs that cannot be transformed into each other by repeated feasible edge-replacements.

\begin{figure}[htbp]
\centering
% Left Subfigure: "U-shape" path
\begin{subfigure}[b]{0.4\textwidth}
    \centering
    \begin{tikzpicture}[scale=2]
        % Define coordinates for a square
        \node[vertex] (TL) at (0,1) {};
        \node[vertex] (TR) at (1,1) {};
        \node[vertex] (BL) at (0,0) {};
        \node[vertex] (BR) at (1,0) {};

        % Draw the path: Left, Top, Right
        \draw[thick] (BL) -- (TL) -- (TR) -- (BR) -- (BL);
    \end{tikzpicture}
\end{subfigure}
\hfill
% Right Subfigure: "Z-shape" / Diagonals
\begin{subfigure}[b]{0.4\textwidth}
    \centering
    \begin{tikzpicture}[scale=2]
        % Define coordinates for a square
        \node[vertex] (TL) at (0,1) {};
        \node[vertex] (TR) at (1,1) {};
        \node[vertex] (BL) at (0,0) {};
        \node[vertex] (BR) at (1,0) {};

        % Draw the path: Bottom-Left to Top-Left, Top-Left to Bottom-Right, Bottom-Right to Top-Right
        % Based on your description: "top edge and the two diagonals"
        \draw[thick] (BL) -- (TR); % Diagonal 1
        \draw[thick] (TR) -- (TL); % Top edge
        \draw[thick] (TL) -- (BR); % Diagonal 2
        \draw[thick] (BL) -- (BR);
    \end{tikzpicture}
\end{subfigure}
\caption{Two $4$-cycles on the same vertex set.}
\label{fig:4_cycles}
\end{figure}

The two paths in Figure~\ref{fig:4_paths} are isomorphic.
I challenge the reader to transform one into the other by a sequence of feasible edge-replacements, or show that it is not possible.

\begin{figure}[htbp]
\centering
% Left Subfigure: "U-shape" path
\begin{subfigure}[b]{0.4\textwidth}
    \centering
    \begin{tikzpicture}[scale=2]
        % Define coordinates for a square
        \node[vertex] (TL) at (0,1) {};
        \node[vertex] (TR) at (1,1) {};
        \node[vertex] (BL) at (0,0) {};
        \node[vertex] (BR) at (1,0) {};

        % Draw the path: Left, Top, Right
        \draw[thick] (BL) -- (TL) -- (TR) -- (BR);
    \end{tikzpicture}
\end{subfigure}
\hfill
% Right Subfigure: "Z-shape" / Diagonals
\begin{subfigure}[b]{0.4\textwidth}
    \centering
    \begin{tikzpicture}[scale=2]
        % Define coordinates for a square
        \node[vertex] (TL) at (0,1) {};
        \node[vertex] (TR) at (1,1) {};
        \node[vertex] (BL) at (0,0) {};
        \node[vertex] (BR) at (1,0) {};

        % Draw the path: Bottom-Left to Top-Left, Top-Left to Bottom-Right, Bottom-Right to Top-Right
        % Based on your description: "top edge and the two diagonals"
        \draw[thick] (BL) -- (TR); % Diagonal 1
        \draw[thick] (TR) -- (TL); % Top edge
        \draw[thick] (TL) -- (BR); % Diagonal 2
    \end{tikzpicture}
\end{subfigure}
\caption{Two paths of length $3$ on the same vertex set.}
\label{fig:4_paths}
\end{figure}

For a set $X$, denote by $S_X$ the set of all bijections (also called \emph{permutations}) $X\to X$.
Endow this set with the topology inherited from the Tychonoff product $X^X$, with $X$ taken as discrete.
Let $G$ be a graph defined on a set $V$ with a labeling $\lambda:V\to X$.
When clear from context, I will conflate a permutation $\sigma\in S_X$ with the induced mapping $V\to V$ given by $\lambda^{-1}\circ\sigma\circ\lambda$.

Suppose that $\overline G$ denotes the graph complement of $G$ and let $(e,f)\in E(G)\times E(\overline G)$.
Write $\sigma:(e\to f)$ to abbreviate the fact that $\sigma$ is a graph isomorphism between $G$ and $G-e+f$.
If $\sigma$ is an automorphism of $G$, write $\sigma:(\emptyset\to\emptyset)$.
Denote by $\Fer(G)$ the group generated by all $\sigma$ for which there are $e$ and $f$ such that $\sigma:(e\to f)$.
These $\sigma$ are called \emph{canonical generators} of $\Fer(G)$.
Canonical generators include automorphisms, and permutations that correspond to moving one edge.
The group $\Fer(G)$ contains elements that can correspond to a sequence of edge movements.
Notice also that the automorphism group $\Aut(G)$, seen as permutations, is a subgroup of $\Fer(G)$, which is itself a subgroup of $S_X$.

It is convenient to represent a sequence of feasible edge-replacements by multiplying their corresponding permutations (which, as standard in Algebra, is written right-to-left) and concatenating the list of edge-replacements (which, as standard in English, is written left-to-right).
There is however a technical caveat: the labels of the graph's vertices change after each canonical generator gets applied (the labels are permuted); but having a list of edge replacements in the graph's original labeling is desirable, for example, if one wishes to execute a solution to a puzzle by purely moving edges without having to keep track of what the corresponding permutations are (not to mention the extra effort needed to figure out what those permutations are in the first place).
For instance, a solution to the puzzle is shown in Figure~\ref{fig:4_cycles_solution} where each caption shows the canonical generator needed to obtain that graph; and the element of $\Fer(G)$ that solves the puzzle (with the same labels as in the figure) is the product of the three generators:
$$(2\ 3):(1\ 2\to1\ 3)(0\ 1\to0\ 2)(2\ 3\to1\ 2).$$
It is worth noting that the permutation $(2\ 3)$ is not a canonical generator, as two edges necessarily change position.

\begin{figure}[htbp]
\centering

% --- STEP 0 ---
\begin{subfigure}[b]{0.21\textwidth}
    \centering
    \begin{tikzpicture}[scale=1.2]
        \node[vertex, label={[vlabel]above left:$v_1$}] (v1) at (0,1) {};
        \node[vertex, label={[vlabel]above right:$v_2$}] (v2) at (1,1) {};
        \node[vertex, label={[vlabel]below left:$v_0$}] (v0) at (0,0) {};
        \node[vertex, label={[vlabel]below right:$v_3$}] (v3) at (1,0) {};
        
        % Add your edges here
        \draw[thick] (v0) -- (v1) -- (v2) -- (v3);
    \end{tikzpicture}
    \caption{Starting graph} \label{step:0}
\end{subfigure}
\hfill $\to$ \hfill
% --- STEP 1 ---
\begin{subfigure}[b]{0.21\textwidth}
    \centering
    \begin{tikzpicture}[scale=1.2]
        \node[vertex, label={[vlabel]above left:$v_1$}] (v1) at (0,1) {};
        \node[vertex, label={[vlabel]above right:$v_3$}] (v3) at (1,1) {};
        \node[vertex, label={[vlabel]below left:$v_0$}] (v0) at (0,0) {};
        \node[vertex, label={[vlabel]below right:$v_2$}] (v2) at (1,0) {};
        
        % Add your edges here
        \draw[thick] (v0) -- (v1) -- (v2) -- (v3);
    \end{tikzpicture}
    \caption{$(2\ 3):(1\ 2\to1\ 3)$} \label{step:1}
\end{subfigure}
\hfill $\rightarrow$ \hfill
% --- STEP 2 ---
\begin{subfigure}[b]{0.21\textwidth}
    \centering
    \begin{tikzpicture}[scale=1.2]
        \node[vertex, label={[vlabel]above left:$v_3$}] (v3) at (0,1) {};
        \node[vertex, label={[vlabel]above right:$v_1$}] (v1) at (1,1) {};
        \node[vertex, label={[vlabel]below left:$v_0$}] (v0) at (0,0) {};
        \node[vertex, label={[vlabel]below right:$v_2$}] (v2) at (1,0) {};
        
        % Add your edges here
        \draw[thick] (v0) -- (v1) -- (v2) -- (v3);
    \end{tikzpicture}
    \caption{$(1\ 3):(0\ 1\to0\ 3)$} \label{step:2}
\end{subfigure}
\hfill $\rightarrow$ \hfill
% --- STEP 3 ---
\begin{subfigure}[b]{0.21\textwidth}
    \centering
    \begin{tikzpicture}[scale=1.2]
        \node[vertex, label={[vlabel]above left:$v_2$}] (v2) at (0,1) {};
        \node[vertex, label={[vlabel]above right:$v_1$}] (v1) at (1,1) {};
        \node[vertex, label={[vlabel]below left:$v_0$}] (v0) at (0,0) {};
        \node[vertex, label={[vlabel]below right:$v_3$}] (v3) at (1,0) {};
        
        % Add your edges here
        \draw[thick] (v0) -- (v1) -- (v2) -- (v3);
    \end{tikzpicture}
    \caption{$(2\ 3):(1\ 2\to1\ 3)$} \label{step:3}
\end{subfigure}

\caption{A sequence of feasible edge-replacements.}
\label{fig:4_cycles_solution}
\end{figure}

For the remainder of the chapter, topological closures are always taken in the containing space $X^X$.
In symbols, if $A\subseteq X^X$, $\overline A:=\cl(X^X,A)$.

\begin{definition}[Local Amoeba]\label{def:local_amoeba}
    A graph $G$ is called a \emph{local amoeba} if $\overline{\Fer(G)}=S_X$.
\end{definition}

Hence, a local amoeba is precisely a graph that can be transformed into any copy of itself on the same vertex set by feasible edge-replacements.
This graph-theoretic notion and the name \emph{amoeba} first appeared in \cite{unavoidable_chromatic} as an interesting combinatorial object with a property that I briefly describe in Subsection~\ref{subsec:amebas_bal}.
When $X$ is finite, all subsets of $X^X$ are closed and thus the preceding definition simplifies to $\Fer(G)=S_X$, which is the algebraic definition introduced in \cite{Caro2020GraphsIU} as a tool to study these graphs and their properties.

If $C_4$ and $P_4$ are the examples depicted in Figure~\ref{fig:4_cycles} and Figure~\ref{fig:4_paths}, respectively, then $\Fer(C_4)=\Aut(C_4)$ is the dihedral group $D_4$ of order $2\cdot4$, and $\Fer(P_4)=S_4$ is the symmetric group.
Therefore, $C_4$ is not a local amoeba, while $P_4$ is.
It should be noted that all of the currently published body of research studies the case when $X$ is finite.

\subsection{Amoebas are balanceable}\label{subsec:amoebas_bal}
\chapter{Descriptive Combinatorics}\label{ch:descriptive_combinatorics}

This is the descriptive combinatorics chapter of the thesis.

\section{Background}\label{sec:descr_combi_background}

This is the place to explain how to get from Cantor's classical results on Polish spaces, to the $G_0$ dichotomy and its applications, to more advanced topics.

\section{A concise proof of the \texorpdfstring{$L_0$}{L0} dichotomy}\label{sec:background}

I present my own proof of the $L_0$ dichotomy.
This approach simplifies the classical arguments and highlights the core combinatorial structures required for the subsequent theorems.

Throughout this entire work, $G$ denotes an analytic graph on a Polish space $V(G)=X$.
That is, $X$ is a separable completely metrizable topological space, and $G$ is a symmetric and irreflexive relation on $X$ which, as a subspace of $X\times X$ is the continuous image of a Polish space.

I denote the Borel chromatic number of $G$ by $\chi_B(G)$, that is, the least number of colours $k$ needed to find a mapping $c:X\to k$ such a way that adjacent vertices in $G$ are given different colours.
I employ the symbol $H\to_cG$ (respectively, $H\to_BG$) to abbreviate the situation where there is a continuous (respectively Borel) homomorphism from the analytic graph $H$ to $G$.
It is straightforward to show the following.

\begin{proposition}\label{fact:chrom_transitive}
    If $H\to_cG$ or, more generally if $H\to_BG$, then $\chi_B(H)\leq\chi_B(G)$.
\end{proposition}

Using a standard greedy algorithm argument, it is clear that finite graphs $G$ of maximum degree $\Delta$ satisfy $\chi(G)\leq\Delta+1$.
Interestingly, a similar fact holds for Borel graphs of uniformly bounded degree.

\begin{theorem}[Proposition~4.6 in \cite{kechris1999borel}]\label{prop:brooks}
    If $G$ is a Borel graph on a Polish space all of whose degrees are bounded by the natural number $k$, then $\chi_B(G)\leq k+1$.
\end{theorem}

\subsection{Finite approximations}

Fix the parameter $c\in\omega^\omega$ and a path of length $\omega$ on the vertex set $\{p_n\}_{n<\omega}$.
That is, $\{(p_n,p_{n+1}):n<\omega\}$.
Naturally, I could assume that $p_n=n$, but this labeling will aid distinguish it from other indices.

I recursively construct a family of graphs $\{L^c_n:n<\omega\}$ such that for all $n<\omega$, the following properties hold.
Take note of the fact that to construct up to $L^c_n$, I only require knowledge of $c\restriction n$.

\begin{enumerate}[label=(\roman*)]
    \item $L^c_n$ is a finite path.
    \item If $n>0$, the endpoints of $L^c_n$ are $e_i^n:=(p_0)^\smallfrown(0)^{n-1}{}^\smallfrown(i)$, for $i<2$.
    \item Every vertex of $L^c_n$ is of the form $(p_k)^\smallfrown t$ for some $k$ and $t\in2^{\leq n}$. Unless $t=\emptyset$, I say the vertex is a \emph{non-path} vertex.
\end{enumerate}

Let $L^c_0=\{(p_0,\dots,p_{c(0)})\}$ and suppose that $L^c_n$ has been constructed.
$L^c_{n+1}$ is a path obtained from joining two copies of $L^c_n$ by a path of length $c(n)$ (meaning, $c(n)$ \emph{edges}), but I must update the vertex labels in order to distinguish the two copies.
I do this by simply appending 0 to each vertex of the first copy, and 1 to the other.
Formally, I start by adding the vertex $v^\frown(i)$ for every $i<2$ and $v\in L^c_n$.
Adding the path $((e_1^n)^\smallfrown(0),p_0,p_1,\dots,p_{c(n)},(e_1^n)^\smallfrown(1))$ completes the construction of $L^c_{n+1}$, and its endpoints are $$e_i^{n+1}:=(e_0^n)^\smallfrown(i)=(p_0)^\smallfrown(0)^n{}^\smallfrown(i),$$ for $i<2$.
This finished the recursion.

\begin{figure}
\caption{$L^c_3$ where $c(n)=2n-1$.}
\label{fig:L_n}
\includegraphics[width=0.8\textwidth]{L0}
\centering
\end{figure}

Understanding the intuition behind what (iii) says about a vertex $v$ in $L^c_n$, the number $|v|-1$ indicates that $v$ was added that many ``recursive stages ago'' at the position of $p_k$ in the path joining the two copies of the previous paths; the sequence $t$ tells the story, in order, of which of the two copies of the previous path the vertex lies in.
In other words, vertices of the form $(p_k)^\smallfrown t$ for $t\in 2^n$ are precisely those that were copied from the original $L^c_0$ in all possible positions along $L^c_n$, doubling the number of copies at each stage, for example.

Based on our discussion, a vertex in some $L^c_n$ is completely determined by the stage $n-m$ at which it was added in the position of $p_k$ and then copied according to $t$.

\subsection{A notion of smallness for copies}

By definition, the set of edges of $G$ is the continuous image of some Polish space $E$, let us say under $\pi$.
For a finite graph $H$, consider $\Hom(H,G)$ the set of all maps

\begin{align*}
    &\varphi:V(H)\to V(G)\\
    &\varphi:E(H)\to E
\end{align*}

that satisfy that for all $uv\in E(H)$, $\pi(\varphi(uv))=\varphi(u)\varphi(v)$.
Formally, $\varphi$ is really two maps, however I assume that $V(H)\cap E(H)=\emptyset$ and so there is never any confusion.
Notice that $\Hom(H,G)$ is a Polish space, being a closed subset of the product space $X^{V(H)}\times(E)^{E(H)}$.

Given $\mathcal H\subseteq\Hom(H,G)$, I define $\mathcal H(u):=\{\varphi(u):\varphi\in\mathcal H\}$, which is analytic whenever $\mathcal H$ is Borel.
Also, $\mathcal H(uv):=\{\varphi(uv):\varphi\in\mathcal H\}$.

\begin{definition}\label{def:small}~ 
    \begin{enumerate}
        \item Let $k<\omega$. I say an analytic set $A\subseteq V(G)$ has property $\Phi(A,k)$ if all odd walks of $G$ with endpoints in $A$ have length at most $2k-1$.

        \item For, $n>0$, a Borel set $\mathcal L\subseteq\Hom(L^c_n,G)$ is called \emph{tiny} if there is a non-path vertex $u\in V(L^c_n)$ and a natural $k$ such that $\Phi(\mathcal L(u),k)$. For $n=0$, $L^c_n$ is just an edge consisting of the two path vertices $p_0$ and $p_1$, so in this case I define $\mathcal L\subseteq\Hom(L^c_0,G)$ as \emph{tiny} if there is a natural $k$ such that $\Phi(\mathcal L(p_1),k)$.

        \item A set $\mathcal L\subseteq\Hom(L,G)$ is \emph{small} if it is the countable union of tiny sets. This notion defines a $\sigma$-ideal.

        \item A set is \emph{large} if it is not small. Notice that when $H$ is the graph of one single vertex, I can define $\Hom(H,G)$ to be $V(G)$ and then for any $\mathcal H\subseteq\Hom(\cdot,G)$, $\mathcal H(\cdot)$ is essentially the same as $\mathcal H$.
    \end{enumerate}
\end{definition}

I need the following auxiliary result that links the 2-colourability and countable chromaticity of a graph in this context.

\begin{lemma}[Claim~3.5 of \cite{L0paper}]\label{lemma:claim3.5}
    If $\Phi(A,n)$, then there is an invariant (that is, union of connected components) Borel set $B\supseteq[A]_{E_G}$ that induces a 2-Borel-colourable subgraph of $G$, that is, there is a Borel proper colouring $c_B:G\restriction B\to2$.
\end{lemma}

\begin{theorem}\label{thm:small}
    If $\chi_B(G)>2$ then $V(G)$ is large.
\end{theorem}

\begin{proof}[Proof of theorem]
    Suppose that $V(G)=\bigcup_{m<\omega}\mathcal H_m$ where each $\mathcal H_m$ is a tiny Borel set.
    By Lemma~\ref{lemma:claim3.5}, there is a $G$-invariant Borel set $B_m\supseteq[\mathcal H_m]_{E_G}$ with a Borel 2-coloring $c_m$.
    Define $c:V(G)\to2$ by letting $m(x):=\min\{m<\omega:x\in B_m\}$ and $c(x)=c_{m(x)}(x)$.
    Then $c$ is a Borel 2-colouring because each $B_m$ is $G$-invariant.
\end{proof}

Since $L^c_{n+1}$ is obtained from two copies of $L^c_n$ joined by a path, I require a way to refer to the distinct copies of $L^c_n$ inside of $L^c_{n+1}$.
By (iii) above, for every $\varphi\in\Hom(L^c_{n+1},G)$, I define $\varphi^i\in\Hom(L^c_n,G)$ on every non-path vertex $u$ of $L^c_n$ by $\varphi^i(u)=\varphi(u^\frown (i))$ for all $u\in V(L^c_n)$ and $\varphi^i(uv)=\varphi(u^\frown(i),v^\frown(i))$ for all $(u,v)\in E(L^c_n)$.
Thus, given $\mathcal L\subseteq\Hom(L^c_n,G)$, define $\mathcal L^{+c}$ as the set of all $\varphi\in\Hom(L^c_{n+1},G)$ for which $\varphi^i\in\mathcal L$ for both $i<2$.
Notice that if $\mathcal L$ is Borel, then so is $\mathcal L^{+c}$.

The length of the path I add needs to be updated dynamically throughout the proof.
Formally, I may need to adjust the parameter $c$ by moving into a different branch of the Baire space when constructing the graphs $L^c_n$.
Notice that, for two reals $c,d\in\omega^\omega$, if $d\restriction n=c\restriction n$, then $L^c_k=L^d_k$ for all $k<n$.

\begin{lemma}\label{lemma:Ln_basics}~ 
    Let $n>0$.
    
    \begin{enumerate}[label=(\alph*)]
        \item Suppose that $c$ only takes odd values, that $k$ is a natural and $t\in2^{<n}$ is such that $(p_k)^\smallfrown t^\smallfrown(i)$, for $i<2$, are non-path vertices of $L^c_n$ (see (iii) above). Then, they are an odd distance apart in $L^c_n$.

        \item If $\mathcal L\subseteq\Hom(L^c_n,G)$ is a large Borel set, then there exists $d\in\omega^\omega$ such that $d\restriction n=c\restriction n$ and $\mathcal L^{+d}$ is a nonempty subset of $\Hom(L^{d}_{n+1},G)$.

        \item If $\mathcal L\subseteq\Hom(L^c_n,G)$ is a large Borel set, then there exists $d\in\omega^\omega$ such that $d\restriction n=c\restriction n$ and $\mathcal L^{+d}$ is a large subset of $\Hom(L^{d}_{n+1},G)$.
    \end{enumerate}
\end{lemma}

\begin{proof}
    For (a), observe that both vertices come from the same vertex in $L^c_{n-1}$, and so their distance in $L^c_n$ is twice the distance to $e_1^{n-1}$ plus the length of the joining path, hence odd.

    (b) follows from the fact that if $\mathcal L$ is large, it is not tiny.
    Hence, $\mathcal L(e_1^n)$ contains the endpoints of a minimal odd path of length $k\geq2n+1>2$, that is, there are $\varphi_0,\varphi_1\in\mathcal L$ and a path in $G$ $(\varphi_0(e_1^n)=v_0,\dots,v_k=\varphi_1(e_1^n))$.
    Let $e_1,\dots,e_k\in E$ be such that $\pi(e_j)=(v_{j-1},v_j)$.
    Defining $d(n)=k-2$ and $d(x)=c(x)$ for all $x\neq n$, notice that, by the remark preceding this lemma, $L^c_n=L^d_n$.
    I find $\varphi\in\Hom(L_{n+1}^d,G)$ be given by $\varphi^0=\varphi_0$, $\varphi^1=\varphi_1$ and $\varphi(p_i)=v_{i+1}$ for $0\leq i\leq k-2=d(n)$; the edges are given by $\varphi(p_j,p_{j+1})=e_j$ for $0\leq j<k-2$, $\varphi({e_1^n}^\frown (0),p_0)=e_1$ and $\varphi(p_{k-2},{e_1^n}^\frown (1))=e_k$.

    I prove (c) by contradiction.
    Recall that constructing $L^c_{n+1}$ from $L^c_n$ requires only knowing the value of $c(n)$.
    Suppose that for all $d$ that are equal to $c$ except in the $n$-th value, $\mathcal L^{+d}=\bigcup_m\mathcal F^d_m$ where each $\mathcal F^d_m$ is a tiny Borel subset of $\Hom(L^d_{n+1},G)$.
    For each $m$ and $d$, let $v^d_m\in V(L^d_n)$, $i^d_m<2$ and $n_d(m)$ be the minimal natural such that $\Phi(\mathcal F^d_m({v^d_m}^\frown(i^d_m)),m_d(n))$.
    
    Now use the Reflection Lemma (reference \ref{lemma:reflection} to be added) to find a Borel $B^d_m\supseteq\mathcal F^d_m(v_m^\frown(i_m))$ such that $\Phi(B^d_m,m_d(n))$.
    Consider $\mathcal H^d_m=\{\varphi\in\mathcal L:\varphi(v^d_m)\in B^d_m\}$, which is a Borel tiny subset of $\Hom(L^d_n,G)=\Hom(L^c_n,G)$.
    Indeed, notice that $\mathcal H^d_m(v^d_m)=\{\varphi(v^d_m):\varphi\in\mathcal H^d_m\}=\{\varphi(v^d_m):\varphi\in\mathcal L\land\varphi(v^d_m)\in B^d_m\}\subseteq B^d_m$.
    
    Consider the large Borel set $\mathcal L_-$ obtained from removing from $\mathcal L$ the countable union of all tiny sets of the form $\mathcal H^d_m$.
    By part (b), there exist $\varphi$ and $d$ such that $\varphi\in\mathcal L_-^{+d}\subseteq\mathcal L^{+d}\subseteq\Hom(L_{n+1}^d,G)$.
    Suppose that $\varphi\in\mathcal F^d_m$, then $\varphi^{i_m^d}(v_m)=\varphi({v_m^d}^\frown(i^d_m))\in\mathcal F^d_m({v_m^d}^\frown(i_m^d))\subseteq B_m$, which means that $\varphi^{i_m^d}\in\mathcal H_m^d\cap\mathcal L_-=\emptyset$; a contradiction.
\end{proof}

\subsection{Construction of a minimal graph of Borel chromatic number at least 3}

The goal of this sections it to construct a family of graphs $\mathbb L_c$ indexed by reals $c\in\omega^\omega$.

Now, fix $c$ and consider $X_c$ as the set of all tuples $(n,k,x)\in\mathbb N\times\mathbb N\times2^\mathbb N$ with $k\leq\max\{1,c(n)\}$ and notice that it is a closed subspace of the product space, and hence Polish.
Define, for each $(m,k,x)$ with $m\leq n$, $\pi_n(m,k,x)$ as the vertex of $L^c_n$ determined by these parameters.
Formally, $\pi_n(m,k,x):=(p_k)^\smallfrown x\restriction(n-m)$.
For example, $\pi_3(1,2,0110\cdots)=(p_2,0)$ can be seen in Figure~\ref{fig:L_n} labeled as $(2,0)$.

Finally, $\mathbb L_c$ is the graph on $X_c$ where two $(n_i,k_i,x_i)$, for $i<2$, are adjacent if and only if the $\pi_n(n_i,k_i,x_i)$ are adjacent in all $L^c_n$ with $n\geq\max\{n_0,n_1\}$.
Some basic properties of $\mathbb L_c$ follow.

\begin{lemma}\label{lemma:L0_basics}~ 
    \begin{enumerate}[label=(\alph*)]
        \item Two vertices $(n_i,k_i,x_i)\in X_c$, for $i<2$, are in the same connected component of $\mathbb L_c$ if and only if there are $t_i\in2^{<\omega}$ and $x\in2^\omega$ such that $|t_0|-|t_1|=n_1-n_0$ and $x_i=t_i^\smallfrown x$.
        
        \item $\mathbb L_c$ has no cycles and all vertices of $\mathbb L_c$ have degree 2 except for the vertex $(0,0,\{0\}^\omega)$, which has degree 1.
        
        \item $\chi_B(\mathbb L_c)=3$.
    \end{enumerate}
\end{lemma}

\begin{proof}
    (c) That $\chi_B(\mathbb L_c)\leq3$ follows from Proposition~\ref{prop:brooks} and the preceding proposition.
    I argue the other inequality by contradiction.
    Suppose then that there is a partition of $X$ into two independent sets.
    Since $X$ is Polish, there is a non-meager class $B$ in the partition.
    $B$ is comeager in some basic open set $[(n,k,t)]$.
    Skipping some technical details, this means that I can find a vertex $(n,k,t^\smallfrown(\varepsilon)^\smallfrown x)\in B$, and moreover not connected to any member of $[(n,k,t)]\setminus B$.
    I claim that $(n,k,t^\smallfrown(1-\varepsilon)^\smallfrown x)\in B$.

    It then follows from Lemma~\ref{lemma:Ln_basics}(a) that these two vertices in $B$ are an odd distance apart, which is impossible.
\end{proof}

The main result is then split into two parts.

\begin{theorem}\label{thm:main}
    For any analytic graph $G$, either $\chi_B(G)\leq 2$ or there is a $c$ such that $\mathbb L_c\to_cG$.
\end{theorem}

Moreover, $c$ can be chosen to be unbounded and take only odd values.

\begin{theorem}\label{thm:L0}
    Let $c,d\in\omega^\omega$ be unbounded and only taking odd values, then $\mathbb L_c$ and $\mathbb L_d$ are continuously homomorphically equivalent.
\end{theorem}

\subsection{Proof of Theorem~\ref{thm:main}}

By Lemma~\ref{lemma:L0_basics}(c) and Proposition~\ref{prop:brooks}, it is clear that both conditions cannot hold simultaneously.
Now suppose that $G$ is an analytic graph of Borel chromatic number at least three.
The theorem is proved by constructing a valid $c$ and a continuous homomorphism witnessing $\mathbb L_c\to_cG$.

By Theorem~\ref{thm:small}, $V(G)$ is large.
I recursively construct a function $c\in\omega^\omega$ and a sequence of large Borel sets $\mathcal L_n\subseteq\Hom(L^c_n, G)$ such that for all $n$,

\begin{enumerate}
    \item $\mathcal L_{n+1}\subseteq\mathcal L_n^{+c}$;
    \item for all $u\in V(L^c_n)$, the diameter of $\overline{\mathcal L_n(u)}$ is less than $2^{-n}$; and
    \item for all $(u,v)\in E(L^c_n)$, the diameter of $\overline{\mathcal L_n(u,v)}$ is less than $2^{-n}$.
\end{enumerate}

By (1), $\mathcal L_{n+1}(\pi_{n+1}(m+2,k,x))\subseteq\mathcal L_n(\pi_n(m,k,x))$ for all $n>m$ and similarly for the edges:
For $(m,k,x)\in X_c$, let $f(m,k,x)$ be the unique point in $$\bigcap_{n=m}^\infty\overline{\mathcal L_n(\pi_n(m,k,x))}=\bigcap_{n=0}^\infty\overline{\mathcal L_{n+m}((p_k)^\frown x\restriction n)}.$$
The map $f:X_c\to X$ is continuous.

It remains to check that it is a homomorphism from $\mathbb L_c$ to $G$.
Suppose that $(m_i,k_i,x_i)_{i<2}$ are adjacent in $\mathbb L_0$, that is, for all $n\geq n_0:=\max\{m_0,m_1\}$, $\pi_n(m_i,k_i,x_i)_{i<2}\in L_n^c$.
There is a unique edge $e\in E$ inside $$\bigcap_{n\geq n_0}\overline{\mathcal L_n(\pi_n(m_0,k_0,x_0),\pi_n(m_1,k_1,x_1))}.$$ By continuity, $(f(m_0,k_0,x_0),f(m_1,k_1,x_1))=\pi(e)\in G$.
Thus the images form an edge of $G$.

\section{A Ramsey-type theorem for \texorpdfstring{$E_0$}{E0} trees}\label{sec:ramsey}

I present a Ramsey-type theorem for $E_0$ trees, which is a key combinatorial ingredient in ongoing work.

\section{Inexistence of a basis for digraphs of dichromatic number 3}\label{sec:dichromatic}

I answer the open question of whether a basis exists for the class of digraphs with dichromatic number at least 3.

\section{Uncountable sets and an infinite linear order game}\label{sec:cantor_game}

This section addresses a question posed by Matthew Baker [Math. Mag. 80 (2007)] regarding the characterization of countable subsets of the reals via infinite games.
While Brian and Clontz recently characterized this for countable payoff sets, the question remained open for general linear orders.

My Contribution: In collaboration with the author of \cite{MatosWiederholdSalvetti2025}, I provided the positive answer to the existence of winning strategies on uncountable sets.
Furthermore, we constructed a dense linear order of size $\kappa$ for every infinite cardinal $\kappa$ on which Player II has a winning strategy.
This construction demonstrates the complexity of generalizing the Brian-Clontz characterization.
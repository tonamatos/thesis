\chapter{Descriptive Combinatorics}\label{ch:descriptive_combinatorics}

This is the descriptive combinatorics chapter of the thesis.

\section{Background}\label{sec:descr_combi_background}

This is the place to explain how to get from Cantor's classical results on Polish spaces, to the $G_0$ dichotomy and its applications, to more advanced topics.

\section{A concise proof of the \texorpdfstring{$L_0$}{L0} dichotomy}\label{sec:background}

I present my own proof of the $L_0$ dichotomy.
This approach simplifies the classical arguments and highlights the core combinatorial structures required for the subsequent theorems.

\section{A Ramsey-type theorem for \texorpdfstring{$E_0$}{E0} trees}\label{sec:ramsey}

I present a Ramsey-type theorem for $E_0$ trees, which is a key combinatorial ingredient in ongoing work.

\section{Inexistence of a basis for digraphs of dichromatic number 3}\label{sec:dichromatic}

I answer the open question of whether a basis exists for the class of digraphs with dichromatic number at least 3.

\section{Uncountable sets and an infinite linear order game}\label{sec:cantor_game}

This section addresses a question posed by Matthew Baker [Math. Mag. 80 (2007)] regarding the characterization of countable subsets of the reals via infinite games.
While Brian and Clontz recently characterized this for countable payoff sets, the question remained open for general linear orders.

My Contribution: In collaboration with the author of \cite{MatosWiederholdSalvetti2025}, I provided the positive answer to the existence of winning strategies on uncountable sets.
Furthermore, we constructed a dense linear order of size $\kappa$ for every infinite cardinal $\kappa$ on which Player II has a winning strategy.
This construction demonstrates the complexity of generalizing the Brian-Clontz characterization.
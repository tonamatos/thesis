\chapter{Group Actions on Graphs}\label{ch:group_actions}

This is the group actions on graphs chapter of the thesis.

\section{Introduction}\label{sec:intro_group_actions}

\section{Border tracing in arbitrary adjacency graphs}\label{sec:border_tracing}

While border tracing is standard for finite planar graphs, I generalize these algorithms to arbitrary adjacency graphs.

My Contribution: I developed and analyzed two generalized algorithms for this setting.

\section{Recursive constructions}\label{sec:recursive_constructions}

I summarize my contributions to both \cite{Eslavaetal2025} and \cite{Caroetal2023} regarding recursive constructions of graphs with interesting combinatorial properties.

\section{Amoeba trees}\label{sec:amoeba_trees}

I present the properties of amoeba trees and recent advancements regarding their conjectures.

My Contribution: I independently conceptualized several key properties discussed in this section.
Additionally, I authored abstractions and object-oriented frameworks that had a significant impact on the results presented in \cite{GomezetalPreprint}.

\section{Infinitary amoeba graphs}\label{sec:infinitary_amoeba_graphs}

I synthesize ideas on infinite graphs, focusing on the interaction between automorphism groups, ends, and amenable groups, specifically placing my own results and ideas concerning amoeba graphs within this broader theoretical context.

\begin{definition}[Local Amoeba]\label{def:local_amoeba}
    A graph $G$ is called a \emph{local amoeba} if $\overline{\Fer(G)}=S_X$.
\end{definition}

Assume that $G$ is a graph defined on an infinite countable set $X$.
In this case, $X^X$ is homeomorphic with the Baire space $\omega^\omega$, and so we are dealing with separable and completely metrizable spaces (i.e.\ Polish spaces).
One can show that $S_X$, being a $G_\delta$ subspace of the Baire space, is also Polish and thus its topology is induced by a complete metric.
Indeed, observe that $$S_X=\bigcap_{\substack{x,y\in X \\ x\neq y}}\{f\in X^X:f(n)\neq f(m)\}\cap\bigcap_{y\in X}\bigcup_{x\in X}\{f\in X^X:f(x)=y\}$$ is a countable (whenever $X$ is) intersection of open sets.
Giving an explicit complete metric is done in the example at the end of the section.

Group actions on graphs may be quite different in the infinite case.
For instance, consider the automorphism of a random graph.
Most finite random graphs are asymmetric but almost all countably infinite random graphs are highly symmetric.

Many properties of finite amoebas translate to our more general definition, such as the fact that $\Fer(G)=\Fer(\overline G)$ for any graph $G$.
Also, any complete graph on any set $X$ is a local amoeba, and thus so is the discrete graph (the graph with no edges).
As a first non-trivial example, the bi-infinite path $P_\mathbb Z$, i.e.\ the graph defined on $X=\mathbb Z$ where the edges are given by consecutive pairs, is interestingly not a local amoeba.
In fact in this context, bi-infinite paths play a similar role as finite cycles.
Indeed, it straightforward to verify that $\Fer(P_\mathbb Z)=\Aut(P_\mathbb Z)$.
Notice that the basic open set $[(0\ 1)(2)]$ is disjoint from $\Fer(G)$ since the only automorphism interchanging 0 and 1 is a reflection (which doesn't fix the element 2), and thus $(0\ 1)\notin\overline{\Fer(G)}$.

\begin{lemma}
    Let $\Gamma\leq S_X$ and $\varphi\in S_X$. Then $\varphi\in\overline{\Gamma}$ if and only if for every finite $F\subseteq X$, there is $\sigma\in\Gamma$ such that $\varphi\restriction F=\sigma\restriction F$.
\end{lemma}

\begin{proof}
    For the forward direction, take $\varphi$ and $F$ as above. Clearly, $[\varphi\restriction F]$ is a basic open neighborhood of $\varphi$ and hence must intersect $\Gamma$. Any $\sigma$ witnessing this has the sought property.
    
    Conversely, take an arbitrary $t\in\omega^{<\omega}$ extended by $\varphi$. Pick $\sigma\in\Gamma$ with $\varphi\restriction \dom t=\sigma\restriction\dom t$. Clearly, $\sigma\in\Gamma\cap[t]$, proving that $\varphi\in\overline{\Gamma}$.
\end{proof}

A group action on $X$ is called \emph{$n$-transitive} if $|X|\geq n$ and for any two pairwise distinct $n$-tuples $(x_1,\dots,x_n)$ and $(y_1,\dots,y_n)$, there is a group element $g$ such that $gx_i=y_i$ for all $1\leq i\leq n$. Notice that $1$-transitive is just the usual definition of \emph{transitive}, i.e.\ that there is a single orbit. If for all $n\geq1$, $\Fer(G)$ acts $n$-transitively on $X$, we say $\Fer(G)$ is \emph{highly transitive}.

\begin{proposition}\label{prop:char_local}
    The following are equivalent.
    \begin{enumerate}
        \item The graph $G$ is a local amoeba.
        \item Every finite $F\subseteq X$ and every $\varphi\in S_X$, there is $\sigma\in\Fer(G)$ such that $\varphi\restriction F=\sigma\restriction F$.
        \item $\Fer(G)$ is highly transitive.
    \end{enumerate}
\end{proposition}

\begin{proof}
    The implication [1$\implies$2] follows from the previous lemma. Assume (2) and let $n\geq1$ and $(x_1,\dots,x_n)$ and $(y_1,\dots,y_n)$ be two pairwise distinct. Let $F:=\{x_i:1\leq i\leq n\}$, then fix a bijection $\omega\setminus F\to\omega\setminus\{y_i:1\leq i\leq n\}$ and define $\varphi$ such that it agrees with said bijection and in addition satisfies $\varphi(x_i)=y_i$. By (2), some $\sigma\in\Fer(G)$ has the desired property.

    Finally, assume (3), let $\varphi$ be an arbitrary permutation on $X$ and $F=\{x_1,\dots,x_n\}\subseteq X$. Apply (3) to the tuples $(x_1,\dots,x_n)$ and $(\varphi(x_1),\dots,\varphi(x_n))$ to get $\sigma\in\Fer(G)$. It follows that $\sigma$ agrees with $\varphi$ on $F$. Since $F$ was arbitrary, by the previous lemma, $G$ must be a local amoeba.
\end{proof}

I point out that in the example of the bi-infinite path, $\Fer(P_\mathbb Z)$ acts transitively on $P_\mathbb Z$ but not 2-transitively, as the same aforementioned example clearly shows, namely the pairs $(0,2)$ and $(1,2)$. In such cases, one could use the largest $n$ for which the group acts $n$-transitively on $G$ to compare how close $G$ is to being a local amoeba.

Denote by $P_{n+1}$ the path on the vertex set $n+1$ with edges given by consecutive pairs of numbers. For a label $i\in X$, denote by $\Fer^i(G)$ the subgroup of $\Fer(G)$ generated by those generators of $\Fer(G)$ that fix $i$. We say a graph $G$ is \emph{stem-symmetric with respect to $i$} (or to the vertex labeled by $i$) if $\Fer^i(G)$ is the symmetric group on $X\setminus\{i\}$.

The next result provides an example of a graph where the gap between the automorphism group and the $\Fer$ group is as large as possible. Concretely, the countably infinite path on $X=\omega$, denoted $P_\omega$ where the edges are given by consecutive pairs of naturals, has a trivial automorphism group, but $\Fer(P_\omega)$ is dense in $S_\omega$.

We need the following combinatorial lemma to prove that infinite paths are local amoebas. The only need the fact that finite paths are stem-symmetric with respect to an endpoint.

\begin{lemma}
    The graph $P_{n+1}$ is stem-symmetric with respect to $k$ if and only if $k\leq n$ and, if $n\geq5$ is odd, $k\neq\frac{n-1}2$.
\end{lemma}

\begin{proof}
    We prove the case $k=n$ as the others are similar. Let $1\leq\ell<n$ and notice that the graph $P_{n+1}-(\ell,\ell+1)+(0,\ell+1)$ is a path on $n+1$ vertices and thus isomorphic to $P_{n+1}$. We can find an explicit isomorphism that fixes $n$ and decompose it into transpositions. In symbols, $$\prod_{k=0}^\ell(k\quad\ell-k):(\ell\ (\ell+1)\to0\ (\ell+1)).$$
    
    The second step is realizing that these permutations indeed generate the symmetric group on (the set) $n$. This is straightforward and can be done in many ways. For example, every transposition of the form $(0\ \ell)$, for $1\leq\ell<n$, is in $\Fer^n(P_{n+1})$ by an inductive argument on $\ell$, taking conjugates and noticing that the transposition in question is the left-most factor in the $\ell$-th permutation above.
\end{proof}

\begin{proposition}
    $P_\omega$ is a local amoeba.
\end{proposition}

\begin{proof}
    Let $F\subseteq\omega$ be finite and $\varphi\in S_\omega$. Define $n:=\max(F\cup\varphi[F])+1$ and notice that $t:=\varphi\restriction F$ is a bijection between finite subsets of $n$. In particular, there exists a bijection $s:n\setminus F\to n\setminus t[F]$. Now define $\sigma:X\to X$ by $$\sigma(x):=\begin{cases}
        s(x)&x\in n\setminus F\\
        t(x)&x\in F\\
        x&x\geq n
    \end{cases}$$ Routine arguments show that $\sigma\in S_X$. Now consider the induced subgraph $P_{n+1}$ and the permutation $\sigma\restriction(n+1)$. Observe that this permutation fixes the endpoint of the path, $n$, by definition. Thus by the previous lemma, there is a sequence $f_0,\dots,f_m$ of generators of $\Fer(P_{n+1})$, all of which fix $n$, such that $\sigma\restriction(n+1)=f_0\cdots f_m$.

    For each $i<m$, we can extend $f_i$ to a permutation $\overline{f_i}$ on $X$ by fixing every label $x>n$. Clearly, each $\overline{f_i}$ is a member of $\Fer(P_\omega)$ and thus $\sigma=\overline{f_0}\cdots\overline{f_m}\in\Fer(P_\omega)$. Finally, the choice of $\sigma$ makes it evident that $\sigma\restriction F=t=\varphi\restriction F$, and thus $P_\omega$ is a local amoeba by Proposition~\ref{prop:char_local}.
\end{proof}
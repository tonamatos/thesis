% /chapters/2-descriptive-combinatorics/05-cantor-game.tex

\section{Uncountable sets and an infinite linear order game}\label{sec:cantor_game}

This section addresses a question by Will Brian and Steven Clontz regarding an infinite game, called \emph{Baker's game}.
In their paper \cite{brian-clontz}, they show that when playing the game in the reals, Player II has a winning strategy if and only if the payoff set is countable.
It makes sense to play this game in more general linear orders, and so they ask if it is possible for Player II to have a winning strategy on an uncountable payoff set instead, or if this characterization somehow only holds in the reals.

In collaboration with the other author of \cite{MatosWiederholdSalvetti2025}, I provided a positive answer to the existence of winning strategies on uncountable sets.
Furthermore, we constructed a dense linear order of arbitrary infinite size on which Player II has a winning strategy on \textbf{every} payoff set.
This construction demonstrates the complexity of generalizing the Brian-Clontz characterization.

\subsection{The Cantor Game}

In \cite{baker}, Matthew Baker introduced the following game on the real numbers:
Fix a subset $S\subseteq\mathbb{R}$.
Player I starts by picking a real number $a_0$.
Then, Player II picks a real number $b_0$ such that $a_0<b_0$.
In the $n$-th turn, Player I picks a real number $a_n$ such that $a_{n-1}<a_n<b_{n-1}$ and then Player II picks a real number $b_n$ such that $a_n<b_n<b_{n-1}$.
After $\omega$-many turns they form two sequences of real numbers $\{a_n\}$ and $\{b_n\}$ such that $a_0<a_1<\cdots<b_1<b_0$.
Player I wins if and only if there exists $x\in S$ such that $a_n<x<b_n$ for all $n<\omega$.
Otherwise, that Player II wins (there are no ties).
Such game can be defined on more general linear orders.
Proposition~\ref{prop:arrow} generalizes the following observation when playing the game in any dense linear order.
  
\begin{proposition}[\cite{baker}]\label{prop:countable win}
    If $S$ is countable, then Player II has a winning strategy in the Cantor Game.
\end{proposition}

The so called \emph{Cantor Game} was introduced as a game-theoretic way of proving that $\mathbb{R}$ is uncountable, which follows by the previous observation.
Baker also proved that if $S$ contains a perfect set, then Player I has a winning strategy.
Later in \cite{cantor}, M.D.\ Ladue (a student of Baker) proved that Player I has a winning strategy if and only if the set $S$ contains a perfect set.
The question of whether the converse of proposition \ref{prop:countable win} is true remained open until 2022.
In \cite{brian-clontz}, Brian and Clontz used elementary submodels to prove the converse statement of proposition \ref{prop:countable win} above.
  
\begin{question}[\cite{brian-clontz}]\label{question:brian-clontz}
    Is there a linear order $(X,<)$ and an uncountable $S\subseteq X$ on which Player II has a winning strategy?
\end{question}

\subsection{Scattered and well-ordered sets}\label{sec:prelim}

Given an ordinal $\alpha$, denote by $\alpha^*$ the converse of $\alpha$, i.e., the conversely well-ordered set $(\alpha,>)$.
Recall that a linearly ordered set is \emph{scattered} if it does not contain a copy of the rationals.
A linearly ordered set is said to be \emph{$\sigma$-scattered} if it is a countable union of scattered sets.
Given an order type $\gamma$, say that a linearly ordered set is \emph{$\gamma$-free} if it does not contain a copy of $\gamma$.
The following result can be found in \cite{partition}.
I include the proof for the reader's convenience.

\begin{theorem}[\cite{todorcevic_6}]\label{thm:todorcevic}
    Let $X$ be a linearly ordered set.
    The following are equivalent:
    \begin{enumerate}[label=(\roman*)]
        \item $X$ is a countable union of well-ordered sets.
        \item $X$ is $\sigma$-scattered and $\omega_1^*$-free.
    \end{enumerate}
\end{theorem}

\begin{proof}
    $(i)\Rightarrow(ii)$ is clear, so I focus on $(ii)\Rightarrow(i)$.
    It suffices to assume that $X$ is scattered (and $\omega_1^*$-free).
    Define the following equivalence relation on $X$: $x\sim y$ if and only if the closed interval $[x,y]$ is a countable union of well-ordered sets.
    Notice that every equivalence class $\mathcal{C}$ is convex, i.e., if $x,y\in\mathcal{C}$ and $x<z<y$, then $z\in\mathcal{C}$.
    Let $[x]$ be the equivalence class of $x$.
    Since every equivalence class is convex, the order $[x]<[y]$ if and only if $x<y$ is well-defined.
    Hence, the quotient $X/\!\sim$ can be seen as a subspace of $X$.
    Thus, $X/\!\sim$ is also scattered by assumption.
    \begin{claim}
        $[x]_{\geq}:=\{y\in[x]:y\geq x\}$ is a countable union of well-ordered sets for all $x$.
    \end{claim}
    \begin{claimproof}
        Let $\{y_\alpha:\alpha<\kappa\}$ be a cofinal subset of $[x]_{\geq}$ with $y_0=x$.
        So, $$[x]_{\geq}=\bigcup_{\alpha<\kappa}[y_\alpha, y_{\alpha+1}].$$
        Since $[y_\alpha, y_{\alpha+1}]$ is a countable union of well-ordered sets, then write $[y_\alpha, y_{\alpha+1}]=\bigcup_{n<\omega}A_n^\alpha$, where $A_n^\alpha$ is $\omega^*$-free.
        Let $A_n:=\bigcup_{\alpha<\kappa}A_n^\alpha$ and notice that $A_n$ is $\omega^*$-free since $\kappa$ is $\omega^*$-free.
        Then, $[x]_\geq=\bigcup_{n<\omega}A_n$, so $[x]_\geq$ is a countable union of well-ordered sets.
    \end{claimproof}
    \begin{claim}
        $[x]_{\leq}:=\{y\in[x]:y\leq x\}$ is a countable union of well-ordered sets for all $x$.
    \end{claim}
    \begin{claimproof}
        Let $\{y_\alpha:\alpha<\kappa\}$ be a coinitial subset of $[x]_\leq$ with $y_0=x$.
        Since $X$ is $\omega_1^*$-free, then $\kappa$ is countable, so
        $$[x]_\leq=\bigcup_{n<\omega}[y_{n+1},y_n].$$
        Since $[y_{n+1},y_n]$ is a countable union of well-ordered sets for all $n<\omega$, then $[x]_{\leq}$ is a countable union of well-ordered sets.
    \end{claimproof}
    The previous two claims imply that $[x]$ is a countable union of well-ordered sets for all $x\in X$.
    Now I show that actually $X=[x]$ for some $x$.
    \begin{claim}
        $(X/\!\sim,<)$ is dense, i.e., if $[x]<[y]$ then there is $z\in Z$ such that $[x]<[z]<[y]$.
    \end{claim}
    \begin{claimproof}
        Let $[x]<[y]$ and suppose there is no $z$ such that $[x]<[z]<[y]$.
        Then, $[x,y]\subseteq [x]_\geq\cup[y]_\leq$ which implies that $[x,y]$ is a countable union of well-ordered sets by the previous two claims.
        Therefore, $[x]=[y]$.
    \end{claimproof}
    This last claim shows that $X$ must have a single equivalence class, since otherwise $X/\!\sim$ would contain a copy of the rationals.
\end{proof}

If $S$ is a countable union of well-ordered subsets of an $\omega_1$-free set $X$, then $S$ must be countable.
Also, it is clear that a subset $S$ of $\mathbb{R}$ is countable if and only if $S$ is a countable union of well-ordered sets.
This idea, in combination with both the previous theorem and the first fact mentioned in this paragraph, is generalized by the following.

\begin{corollary}\label{cor:sigma-countable}
    If $X$ is an $\omega_1$-free and $\omega_1^*$-free linearly ordered set, then the following are equivalent for any $S\subseteq X$.
    \begin{enumerate}
        \item $S$ is countable.
        \item $S$ is a countable union of well-ordered sets.
        \item $S$ is $\sigma$-scattered.
    \end{enumerate}
\end{corollary}

\subsection{Baker's game on linear orders}\label{sec:linear}
I define a variation of Baker's game in an arbitrary dense linear order $(X,<)$.
Given $S\subseteq X$, denote by $BG(X,S)$ the game where two players take turns selecting elements $a_n,b_n$ of $X$ as in the Cantor Game for the reals.
Player I wins if after $\omega$-many turns there is an $x\in S$ such that for all $n<\omega$, $a_n<x<b_n$, and Player II wins otherwise.
The density condition ensures the game never reaches a stalemate.
Thus, for the rest of the section, $X$ is a dense linearly ordered set with a fixed subset $S$ (the winning conditions).
Observe that every countable set $S$ satisfies the subsequent proposition.

\begin{proposition}\label{prop:arrow}
    If $S$ is a countable union of well-ordered sets, then Player II has a winning strategy for $BG(X,S)$.
\end{proposition}

\begin{proof}
    Let $S=\bigcup_{n<\omega}S_n$ where each $S_n$ is $\omega^*$-free (that is, well-founded).
    In the $n$-th turn, Player II plays $b_n=\min(S_n\cap(a_n,b_{n-1}))$ if $S_n\cap(a_n,b_{n-1})\neq\emptyset$ and any legal move otherwise (by density, there must always be one).
    Now, suppose that for all $n<\omega$, $a_n<x<b_n$.
    This implies that $x\notin S_n$ for all $n<\omega$, and hence, $x\notin S$.
    In other words, this is a winning strategy for Player II.
\end{proof}

In general, $S$ being a countable union of well-ordered sets does not imply that $S$ is countable (unless it contains no copies of $\omega_1$).
The previous proposition implies that if $S$ is an (uncountable) well-ordered set, then Player II has a winning strategy for $BG(X,S)$.
I now show that if $S$ is conversely well-ordered, then Player II has a winning strategy for $BG(X,S)$.
Notice that an uncountable conversely well-ordered set $S$ has the property that $S$ cannot be written as a countable union of well-ordered sets, so this result shows that, in general, the converse of Proposition ~\ref{prop:arrow} is false.

\begin{theorem}\label{thm:blocks}
    Suppose that $\gamma$ is an ordinal and $\{S_\alpha:\alpha<\gamma\}$ is a collection of subsets of $X$ on which Player II has a winning strategy.
    Furthermore, assume that $S_\beta<S_\alpha$ for all $\alpha<\beta<\gamma$ (i.e.\, $x<y$ for all $x\in S_\beta$ and $y\in S_\alpha$).
    Then, Player II has a winning strategy for $BG\left(X,\bigcup_{\alpha<\gamma}S_\alpha\right)$.
\end{theorem}

\begin{proof}
    Player I starts by playing $a_0\in X$.
    If there is $b_0>a_0$ such that $b_0\leq y$ for all $y\in\bigcup_{\alpha<\gamma}S_\alpha$, then Player II wins by picking $b_0$.
    Otherwise, Player II picks any legal move in the first turn.
    In this case, $a_1>a_0$, so let $\alpha_1<\gamma$ be the smallest such that $y_1<a_1$ for some $y_1\in S_{\alpha_1}$.
    If there is a legal move $b_1>a_1$ such that $b_1\leq y$ for all $y\in\bigcup_{\alpha<\alpha_1}S_\alpha$, then Player II picks such $b_1$ and then uses the winning strategy for $S_{\alpha_1}$.
    Otherwise, Player II plays any legal move $b_1$ and then, since $a_2>a_1$, one can pick the smallest $\alpha_2<\alpha_1$ such that $y_{2}<a_2$ for some $y_2\in S_{\alpha_2}$.
    
    By repeating this process, since $\gamma$ is well-ordered, one cannot have a decreasing sequence $\dots<\alpha_2<\alpha_1<\gamma$.
    Hence, at some turn $N$, there is a legal move $b_N>a_N$ for Player II such that $b_N\leq y$ for all $y\in\bigcup_{\alpha<\alpha_N}S_\alpha$.
    After that, Player II continues playing with the winning strategy for $BG(X,S_{\alpha_N})$.
    
    This describes a winning strategy for Player II.
    Indeed, given any $x\in X$ such that $a_n<x<b_n$ for all $n<\omega$, the following holds:
    Given that $y_N<a_N<x$ and the hypothesis that $S_\beta<S_\alpha$ whenever $\alpha<\beta$, $x\notin S_{\alpha}$ for all $\alpha>\alpha_N$; also $x\notin S_{\alpha}$ for all $\alpha<\alpha_N$ by the way $b_N$ was played.
    Finally, $x\notin S_{\alpha_N}$, since Player II used a winning strategy for $BG(X,S_{\alpha_N})$.
\end{proof}

\begin{corollary}
    If $S$ is conversely well-ordered, then Player II has a winning strategy on $BG(X,S)$.
\end{corollary}

\begin{corollary}\label{cor:any_kappa}
    For any infinite cardinal $\kappa$, there is a dense linear order $(X,<)$ of size $\kappa$ on which Player II has a winning strategy on every subset of $X$.
\end{corollary}

\begin{proof}
    Take $X=\kappa^*\times\mathbb Q$ ordered lexicographically.
    Concretely, given $x=(\alpha,p)$ and $y=(\beta, q)$ elements of $X$, $x\leq y$ if either $\beta<\alpha$ or $\alpha=\beta$ and $p<_\mathbb Qq$.
    Since the rationals are countable and $\kappa$ is infinite, clearly $X$ has the desired size.
    The fact that $X$ is dense follows from the fact that $\mathbb Q$ is dense and unbounded.

    Now let $S\subseteq X$ and define, for each $\alpha<\kappa$, $S_\alpha=\{\alpha\}\times\mathbb Q$.
    Clearly, each $S_\alpha$ is countable, and thus Player II has a winning strategy for $BG(X,S_\alpha)$, by virtue of Proposition~\ref{prop:arrow}.
    Since $S=\bigcup_{\alpha<\kappa}S_\alpha$ and the $S_\alpha$ are ordered the way that Theorem~\ref{thm:blocks} requires, the result follows.
\end{proof}

\subsection{Concluding remarks and open questions}

When does the converse of Proposition~\ref{prop:arrow} hold?
By the previous results such orderings must be $\omega_1$-free and $\omega_1^*$-free.
Separable and even ccc linear orderings have this property.
The separable case (the reals) was established in \cite{brian-clontz}.
Then, the next step is to study dense linear orders that are $\omega_1$-free, $\omega_1^*$-free but not separable.
The typical example of such ordering is the Aronszajn ordering.

\begin{definition}
    An \emph{Aronszajn line (A-line)} is an uncountable linearly ordered set that is $\omega_1$-free, $\omega_1^*$-free and contains no uncountable subset of the reals.
\end{definition}

Following \cite{A-orderings}, define a proper decomposition for A-lines.
Let $A$ be a dense Aronszajn line.
Let $D_0:=\emptyset$.
Given a countable $D_\alpha\subseteq A$, consider the equivalence relation on $A\backslash D_\alpha$: $x\sim_\alpha y$ if and only if there is no $z\in D_\alpha$ such that $x<z<y$.
Notice that each $\sim_\alpha$ class is convex.
Let $T_\alpha$ be the set of all $\sim_\alpha$ classes and let $D_\alpha'$ be a selector of the $\sim_\alpha$ classes.
Since $D_\alpha$ is countable and dense in $D_{\alpha+1}:=D_\alpha\cup D_\alpha'$, then $D_{\alpha+1}$ must be countable because $A$ does not contain uncountable separable subsets.
Given $D_\alpha$ for $\alpha<\gamma$ and $\gamma$ limit, let $D_\gamma=\bigcup_{\alpha<\gamma}D_\alpha$.
Since $A$ does not contain copies of $\omega_1$ and $\omega_1^*$, $A=\bigcup_{\alpha<\omega_1}D_\alpha$.
Also, $(T,\supseteq)$ is an Aronszajn tree where $T=\bigcup_{\alpha<\omega_1}T_\alpha$.

\begin{question}
    Is there a dense Aronszajn line $A$ where Player II has a winning strategy in $BG(A,S)$ for all $S\subseteq A$?
\end{question}

Another class of lines where a similar decomposition can be performed is the class of nowhere separable Souslin continua.

\begin{definition}
    A \emph{Souslin continuum} is a non-separable dense, complete linearly ordered set that has the countable chain condition (ccc), i.e., every family of pairwise disjoint open intervals is countable.
    A Souslin continuum is \emph{nowhere separable} if no non-empty open interval is separable.
\end{definition}

Define a \emph{proper decomposition} for the nowhere separable Souslin continuum $L$ as follows.
let $D_0:=D\neq\emptyset$ be any countable set.
Given a countable $D_\alpha\subseteq L$ consider the equivalence relation on $L\backslash\overline{D_\alpha}$:
$x\sim_\alpha y$ if and only if there is no $z\in D_\alpha$ such that $x<z<y$.
Notice that each $\sim_\alpha$ class is convex.
Let $T_\alpha$ be the set of all $\sim_\alpha$ classes and let $D_\alpha'$ be a selector of the $\sim_\alpha$ classes.
Since $L$ has the ccc, then $T_\alpha$ is countable.
Let $D_{\alpha+1}=D_\alpha\cup D_\alpha'$.
For $\gamma<\omega_1$ limit let $D_\gamma=\bigcup_{\alpha<\gamma}D_\alpha$.

\begin{lemma}
    If $\{D_\alpha:\alpha<\omega_1\}$ is a proper decomposition of a nowhere separable Souslin continuum $L$, then $$L=\bigcup_{\alpha<\omega_1}\overline{D_\alpha}.$$
\end{lemma}

\begin{proof}
    Suppose that $x\in L$ and $x\notin \bigcup_{\alpha<\omega_1}\overline{D_\alpha}$.
    For each $\xi<\omega_1$ let $I_\xi^0:=[x]_\xi$ (the $\sim_\xi$ class of $x$).
    Note that $I_\eta^0\subseteq I_\xi^0$ for all $\xi<\eta$.
    \begin{claim}
        For all $\alpha<\omega_1$, $|T_\alpha|>1$.
    \end{claim}
    \begin{claimproof}
        Let $x_0\in D_\alpha$.
        Since $L$ is nowhere separable, $\overline{D_\alpha}$ is nowhere-dense so pick $y_1\in (L\backslash\overline{D_\alpha})\cap (-\infty, x_0)$ and $y_1\in (L\backslash\overline{D_\alpha})\cap (x_0,\infty)$.
        Then, $[y_1]_\alpha,[y_2]_\alpha\in T_\alpha$.
    \end{claimproof}
    For each $\xi<\omega_1$ let $I_{\xi+1}^1\in T_{\xi+1}$ such that $I_{\xi+1}^1\cap I_{\xi+1}^0=\emptyset$.
    Then, $\{I_{\xi+1}^1:\xi<\omega_1\}$ is an uncountable pairwise disjoint family of open sets, contradicting ccc.
\end{proof}

Using a proper decomposition to show the following remains an open question of great interests to the authors of \cite{MatosWiederholdSalvetti2025}.

\begin{conjecture}
    Let $L$ be a nowhere separable Souslin continuum.
    Then Player II has a winning strategy for $BG(L,S)$ if and only if $S$ is countable.
\end{conjecture}